\documentclass{article}

\usepackage[bitwidth=auto]{bytefield}

\title{Reach Network Communication Protocol}
\author{Ian Dewan}
\date{20 February, \textsc{62 Eliz. II}}

\def\tname{\texttt}
\def\literal{\texttt}
\newcommand{\tbb}[2]{\bitbox{#1}{\tname{#2}}}
\newcommand{\twb}[2]{\wordbox{#1}{\tname{#2}}}

\begin{document}
\maketitle
This file describes the format used to communicate between networked buzzers and
the Reach for the Top program.

All communication is by UDP datagrams; all datagrams are exactly 12 bytes long.
All multi-byte quantities are in network byte ordering. After a datagram is
received, the receiving program must immediately send a confirmation datagram
(see below) to the sender. If confirmation is not received within an arbitrary
time limit, the sender may retransmit the datagram; this may be repeated any
number of times.

Every packet has a unique packet ID, a 16 bit number. Odd numbers are reserved
for the client (buzzer/phone) and even for the server. Packet IDs may never be
reused except in the following circumstances:
 \begin{itemize}
 \item Confirmation datagrams \emph{must} have the same packet ID as the
  datagram they confirm.
 \item Retransmitted datagrams \emph{must} have the same ID as the original
  datagram.
 \end{itemize}
If a datagram is received with an already used packet ID, it must be ignored.

In the following datagram descriptions, \tname{RESERVED} means that the bits are
not currently used, but may be in later versions of the standard:
implementations \emph{must} ignore the values of these bits. \tname{UNUSED}
means that the bits will never be used in a future standard.

\section{Datagram Formats}

\subsection{The \tname{CONFIRM} datagrams}
\begin{bytefield}{32}
\bitheader{0-31}\\
\tbb{8}{TYPE} & \tbb{8}{FLAGS} & \tbb{16}{PACKET ID} \\
\twb{2}{UNUSED}
\end{bytefield}

Sent by either server or client.
\begin{description}
\item[\tname{TYPE}] always \literal{0xC0} for \tname{CONFIRM} datagrams.
\item[\tname{FLAGS}] see below
\item[\tname{PACKET ID}] same as the datagram being confirmed.
\end{description}

\subsection{The \tname{JOIN} datagram}
\begin{bytefield}{32}
\bitheader{0-31}\\
\tbb{8}{TYPE} & \tbb{8}{FLAGS} & \tbb{16}{PACKET ID} \\
\tbb{8}{TEAM} & \tbb{24}{RESERVED} \\
\twb{1}{RESERVED}
\end{bytefield}

Sent by client.
\begin{description}
\item[\tname{TYPE}] always \literal{0x07} for \tname{JOIN} datagrams.
\item[\tname{FLAGS}] see below
\item[\tname{PACKET ID}] see above discussion
\item[\tname{TEAM}] the (zero-based) index of the team one wishes to join.
\end{description}
The server must respond to the \tname{JOIN} datagram with a
\tname{JOIN\_RESPONSE} datagram.

\subsection{The \tname{JOIN\_RESPONSE} datagram}
\begin{bytefield}{32}
\bitheader{0-31}\\
\tbb{8}{TYPE} & \tbb{8}{FLAGS} & \tbb{16}{PACKET ID} \\
\tbb{16}{RESPONSE TO} &\tbb{8}{ERROR} & \tbb{2}{H} & \tbb{6}{RESERVED} \\
\twb{1}{RESERVED}
\end{bytefield}

Sent by server.
\begin{description}
\item[\tname{TYPE}] always \literal{0x97} for \tname{JOIN\_RESPONSE} datagrams.
\item[\tname{FLAGS}] see below
\item[\tname{PACKET ID}] see above discussion
\item[\tname{RESPONSE TO}] The packet id of the \tname{JOIN} packet this is a
 response to.
\item[\tname{ERROR}] The error code, if an error occurred, or zero for success.
 Valid error codes are:
 \begin{enumerate}
 \item Invalid team number
 \item Team specified is already full.
 \end{enumerate}
\item[\tname{H}] The handset number (position) on the joined team, if joining
 succeeded.
\end{description}

\subsection{The \tname{BUZZ} datagram}
\begin{bytefield}{32}
\bitheader{0-31}\\
\tbb{8}{TYPE} & \tbb{8}{FLAGS} & \tbb{16}{PACKET ID} \\
\twb{2}{RESERVED}
\end{bytefield}

Sent by client.
\begin{description}
\item[\tname{TYPE}] always \literal{0xB2} for \tname{JOIN} datagrams.
\item[\tname{FLAGS}] see below
\item[\tname{PACKET ID}] see above discussion
\end{description}

\subsection{The \tname{STATE} datagram}
\begin{bytefield}{32}
\bitheader{0-31}\\
\tbb{8}{TYPE} & \tbb{8}{FLAGS} & \tbb{16}{PACKET ID} \\
\tbb{1}{L} & \tbb{1}{B} & \tbb{30}{RESERVED} \\
\twb{1}{RESERVED}
\end{bytefield}

Sent by server.
\begin{description}
\item[\tname{TYPE}] always \literal{0x5A} for \tname{STATE} datagrams.
\item[\tname{FLAGS}] see below
\item[\tname{PACKET ID}] see above discussion
\item[\tname{L}] 1 if the light should be on, 0 for off.
\item[\tname{B}] 1 if the client should stop sending buzz events, 0 if it may
 send buzz events.
\end{description}

\section{The contents of the \tname{FLAGS} byte}
\begin{bytefield}{8}
\bitheader{0-7}\\
\tbb{1}{NC} & \tbb{7}{RESERVED}
\end{bytefield}

\begin{description}
\item[\tname{NC}] if this bit is set, the sending of a \tname{CONFIRM} datagram
 for this datagram is optional.
\end{description}
\end{document}
